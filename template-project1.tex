\documentclass{article}
\usepackage[utf8]{inputenc}

\usepackage{amsfonts}
\usepackage{amssymb}
\usepackage{amsmath}
\usepackage{amsthm}
\usepackage{enumitem}

\usepackage{bbold}
\usepackage{bm}
\usepackage{graphicx}
\usepackage{color}
\usepackage{hyperref}
\usepackage[margin=3cm]{geometry}

\begin{document}


% ==============================================================================

\title{\Large{INFO8006: Project 1 - Report}}
\vspace{1cm}
\author{\small{\bf Maxime Goffart - s180521} \\ \small{\bf Olivier Joris - s182113}}

\maketitle

% ==============================================================================

\section{Problem statement}

\begin{enumerate}[label=\alph*.,leftmargin=1.35em]
    \item
    	\begin{itemize}
    		\item The initial state is given by the layout of the maze which gives the initial position of Pacman and the initials positions of all the food dots. The possible states are a combination of Pacman position and the positions of the remaining food dots.
    		\item The legal actions are going north, going south, going west, and going east as long as Pacman does not go through a wall.
    		\item The action taken will have the effect to change Pacman position in the maze (if the action is legal). If the cell on which Pacman arrives contains a food dot, the food dot is eaten and removed from the maze.
    		\item The goal is reached if all the food dots have been eaten.
    		\item The step cost can be defined as follows: each move of Pacman to a cell without a food dot costs 10 and each move to a cell with a food dot costs 1.
    	\end{itemize}
\end{enumerate}

\section{Implementation}

\begin{enumerate}[label=\alph*.,leftmargin=1.35em]
    \item The error lies in the \textit{key} function and is the fact that it identifies a game state by taking only the position of Pacman. However, to identify the state of the game, the position of the food dots is also needed. Indeed, if Pacman has 2 positions that are the same, it may have eaten food dots in the meantime: the game state is then no longer the same.\\
          This error can be corrected by returning a tuple containing both the current position of Pacman and the position of the food dots in the \textit{key} function.  
    \item \textbf{{\it Leave empty.}}
    \item
    \item
    \item \textbf{{\it Leave empty.}}
    \item
\end{enumerate}

\section{Experiment 1}

\begin{enumerate}[label=\alph*.,leftmargin=1.35em]
    \item
    \item
    \item
    \item
\end{enumerate}

\section{Experiment 2}

\begin{enumerate}[label=\alph*.,leftmargin=1.35em]
    \item
    \item
    \item
\end{enumerate}


% ==============================================================================

\end{document}