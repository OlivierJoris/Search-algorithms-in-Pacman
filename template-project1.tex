\documentclass{article}
\usepackage[utf8]{inputenc}

\usepackage{amsfonts}
\usepackage{amssymb}
\usepackage{amsmath}
\usepackage{amsthm}
\usepackage{enumitem}

\usepackage{bbold}
\usepackage{bm}
\usepackage{graphicx}
\usepackage{color}
\usepackage{hyperref}

\begin{document}


% ==============================================================================

\title{\Large{INFO8006: Project 1 - Report}}
\vspace{1cm}
\author{\small{\bf Olivier Joris - s182113} \\ \small{\bf Maxime Goffart - s180521}}

\maketitle

% ==============================================================================

\section{Problem statement}

\begin{enumerate}[label=\alph*.]
    \item 
       \begin{itemize}
            \item[\textbullet] The set of possible states is defined by a matrix representing the map in which for each element there is a boolean value
                               indicating the presence of a food dot and another indicating the presence of Pac-Man in this cell of the map.\\
                               The initial state is set by the layout.
            \item[\textbullet] At any moment, Pac-Man can move up, down, right or left as long as he does not go through a wall (illegal action).
            \item[\textbullet] When Pac-Man performs one of the previous legal actions, some elements of the previously defined matrix are then updated according to the current state of the game.
            \item[\textbullet] The goal is reached when the elements of the matrix indicating the presence of a food dots are all at the 'False' value : there are no food dots on the map.
            \item[\textbullet] The step cost can be defined as follows : each move of Pac-Man to a cell without a food dot costs 10 and each move to a cell with a food dot costs 1.
       \end{itemize}
\end{enumerate}

\section{Implementation}

\begin{enumerate}[label=\alph*.]
    \item
    \item \textbf{LEAVE EMPTY}
    \item
    \item
    \item \textbf{LEAVE EMPTY}
    \item
\end{enumerate}

\section{Experiment 1}

\begin{enumerate}[label=\alph*.]
    \item
    \item
    \item
    \item
\end{enumerate}

\section{Experiment 2}

\begin{enumerate}[label=\alph*.]
    \item
    \item
    \item
\end{enumerate}


% ==============================================================================

\end{document}
